%#Author = Shalabodov Yaroslav Dmitrievich
%#Group = P3110
%#Date =  23.11.25
% \documentclass[11pt]{article}
% \usepackage[english,russian]{babel}
% \usepackage[a4paper, margin=0.5cm, bottom=1.5cm]{geometry}
% \usepackage{microtype}
% \usepackage{ragged2e} % Улучшенное выравнивание текста
% \usepackage{graphicx} % Для добавления иллюстраций
% \usepackage{tabularx} % Таблицы
% \usepackage{icomma} % Делает запятую в формулах более "интеллектуальной"
% \usepackage{lipsum} 
% \usepackage{amssymb}
% \usepackage{paracol} % Разбиение текста на колонки
% \usepackage{amsmath} % Математические формулы
% \usepackage{fancyhdr} % Настройка стиля страницы
% \usepackage{pifont}
% \usepackage{xcolor}
% \usepackage{gensymb}
% \pagecolor[rgb]{0.95,0.93,0.86}
% \graphicspath{{./images}} % Путь до картинок
% \columnratio{0.33,0.67} % Соотношение размеров колонок
% \setlength{\emergencystretch}{5em} % Максимальное расстояние между словами


% \begin{document}

    \pagestyle{fancy} % Стиль страницы с верхним и нижним колонтитулами
    \fancyhead{} % Очистить верхний колонтитул
    \fancyfoot{} % Очистить нижний колонтитул
    \pagenumbering{arabic} % Нумерация страниц арабскими цифрами
    \fancyfoot[L]{\textbf{\thepage}} % Номер страницы в левом части нижнего колонтитула
    \setcounter{page}{34} % Установить номер страницы
    \setcounter{figure}{4} % Установить номер картинки
    
    \begin{paracol}{2}
        \begin{column}
            \vspace*{\fill}
            \begin{figure}[h]
                \centering
                \includegraphics[width=0.8\linewidth]{sphere.png}
                \label{fig:sphere}
            \end{figure}
            \raggedright
            \textbf{Рис. 5.}\\
            \begin{figure}[h]
                \centering
                \includegraphics[width=0.8\linewidth]{sphere-with-plane.png}
                \label{fig:sphere-with-plane}
            \end{figure}
            \raggedright
            \textbf{Рис. 6.}\\
            \vspace*{\fill}
        \end{column}

        \begin{column}
            \noindent Этот ответ у нас получился так:\\
            \\
            \begin{tabularx}{\linewidth}{
            |>{\hsize=1.8\hsize\raggedright\arraybackslash}X
            |>{\hsize=0.73\hsize\centering\arraybackslash}X
            |>{\hsize=0.73\hsize\centering\arraybackslash}X
            |>{\hsize=0.73\hsize\centering\arraybackslash}X|}
                \hline 
                    &5-уголь-ников&4-уголь-ников&3-уголь-ников\\
                \hline
                    &&&\\
                    4 круга разбивают сферу на&0&6&8\\
                    &&&\\
                \hline
                    &&&\\
                    5-й круг оставляет нетрону-\ тыми&0&2&4\\
                    разбивает 3-угольники на&0&4&4\\
                    разбивает 4-угольники на&2&4=2+2&2\\
                    &&&\\
                \hline
                    &&&\\
                    \rightline{\textls[350]{Итого  \ldots}} & 2 & 10 & 10 \\
                    &&&\\
                \hline
            \end{tabularx} \\
            \\
            
            А теперь покажем, как этот же ответ получить намного проще.
            
            \begingroup
            \setlength{\parindent}{0pt}
            \textls[350]{Второе решение}. Спроектируем сферу из центра $P$ на плоскость $\alpha$, параллельную э\textbf{к}ватору $O$. Тогда пара симметричных $n$-угольников на сфере, не примыкающих к $O$, отобразится на $n$-угольник в плоскости $\alpha$, а пара $n$-угольников, примыкающих к $O$ --- на $(n-2)$-угол; пользуясь результатом задачи a) --- единственностью разбиения плоскости четырьмя прямыми и первой табличкой, -- мы заключаем, что любые пять больших кругов разбивают сферу на два пятиугольника (их проекция--3-угол), 2(1+4)=10 четырехугольников (из них восемь примыкают к данному большому кругу $O$, и эти \textit{четыре пары проектируются на 2-углы}), и 2(2+3)=10 треугольников (из них шесть примыкают к большому кругу $O$). Те, кто знаком с понятием проективной плоскости (см., например, «Квант», 1974, №3), заметят, конечно, что «проекция», о которой мы говорим, более естественно объясняется как отображение сферы на \textit{проективную плоскость} (при \textbf{это}м отображении две диаметрально противоположные точки сферы склеиваются в одну, образы всех больших кругов — прямые, в том числе образ круга $O$—\textit{бесконечно удалённая прямая}).
            
            Рисунок \ref{fig:sphere-with-plane}, иллюстрирующий второе решение, нужно пояснить. Поскольку «бесконечно удалённую прямую» нарисовать трудно, мы на рисунке 6 очень близко к большому кругу—экватору $O$ — провели пунктиром п\textbf{а}раллель; ее образом при \textit{проекции будет окружность очень большого} радиуса в плоскости $\alpha$, содержащая внутри себя все точки пересечения четырёх прямых; эти прямые раз\textbf{б}ивают возникающий на плоскости «очень большой круг» на 11 областей: «пятиугольников», пять «треугольников» и пять «четырехугольников»; возвращаясь с плоскости на сферу, получаем после удвоения отве\textbf{т}. В заключение за\textbf{м}етим, что во всех трёх задачах не только вид частей разбиения (число сторон), но и их взаимное расположение определяются однозначно.
            
            Предлагаем читателям «Кванта» в качестве задачи для исследования выяснить, \textbf{к}акие бывают разбиения сферы шестью большими кругами или плоскости — пятью прямыми (здесь уже нет единственности). Самое трудное здесь, по-видимому, доказать то, что найдены все возможные варианты разбиения.
            \endgroup
            
            \rightline{\textit{Н. Васильев}} 
            \vspace{0.5cm}
        \end{column}
    \end{paracol}

    \begin{paracol}{2}
        \begin{column}
        \end{column}
        \begin{column}
            \noindent\LARGE\ding{117}
            \vspace{0.5cm}
        \end{column}
    \end{paracol}

    \begin{paracol}{2}
        \begin{column}
            \noindent\textbf{M445}. \textit{Центры одинаковых непересекающихся окружностей находятся в центрах правильных шестиугольников, покрывающих плоскость так, как показано на рисунке 7. Пусть М-многоугольник с вершинами в центрах окуржностей. Окрасим в красный цвет те окружности или их части (дуги), которые лежат внутри М. Покажите, что сумма градусов равна C$\cdot$180$\degree$, С$\approx$С(М) -- целое число, и дайте этому числу геометрическую интерпертацию.}\\
            
            \begin{figure}[h]
                \centering
                \includegraphics[width=0.8\linewidth]{images/figure.png}
            \end{figure}
            \raggedright
            \textbf{Рис. 7.}
        \end{column}

        \begin{column}
            \noindentЗаметим, что центры правильныъ шестиугольников находятся на вершинах ромбов (сторона ромба -- удвоенная апофема шестиугольника, угол при вершине равен 60$\degree$). Таким образом, мы имеем \textit{ромбическую решётку} на плоскости; вершины ромбов -- \textit{узлы} решётки (см. рис. 7). Далее, если многоугольник М с вершинами в узлах представлен в виде объединения многоугольников $A_1, \ldots, A_k$ (с вершинами в узлах): $M=\bigcup\limits_{i=l}^{k} A_i$, то
            \[ 
            C(M) = C(A_1)+\ldots+C(A_k);
            \]
            \noindent про аткую функцию говорят, что она \textit{аддитивна}. Отсюда следует, что если многоугольник М с вершинами в узлах представлен в виде объединения многоугольников $A_1, \ldots, A_k$, из которого удалено объединение многоугольников $B_1, \ldots, B_l$ ($A_i$ и $B_j$ --- с вершинами в узлах) --- см. рисунок 8; т.е. если 
            \begin{equation}
                \tag{*} M=\bigcup\limits_{i=1}^{k} A_i \setminus \bigcup\limits_{j=1}^{l} B_j \text{, причем }\bigcup B_j \subset \bigcup A_i
            \end{equation}
            \noindent (здесь $\smallsetminus$ --- значок \textit{разности множеств} или \textit{дополнения}), то
            \[
            C(M)=\sum\limits_{i=1}^{k} C(A_i)-\sum\limits_{j=1}^{l} C(B_j).
            \]
        \end{column}
    \end{paracol}
    \clearpage
% \end{document}